\documentclass{article}
\usepackage{amsmath}
\usepackage{algorithm}
\usepackage{algorithmic}

\title{LaTeX 数学公式示例文档}
\author{示例作者}
\date{\today}

\begin{document}

\maketitle

\section{引言}

这是一个包含各种 LaTeX 数学公式的示例文档。本文档展示了行内公式、块级公式、矩阵、积分等常见数学表达式的渲染效果。

\section{基本数学公式}

\subsection{行内公式}

行内公式使用美元符号表示,例如:$E = mc^2$ 是爱因斯坦的质能方程。另一个例子是勾股定理:$a^2 + b^2 = c^2$。

\subsection{块级公式}

块级公式使用双美元符号或方括号表示:

$$F = G \frac{m_1 m_2}{r^2}$$

\[\sum_{i=1}^n i = \frac{n(n+1)}{2}\]

\section{复杂数学表达式}

\subsection{积分和微分}

不定积分:$$\int x^2 dx = \frac{x^3}{3} + C$$

定积分:$$\int_0^1 x^2 dx = \left[\frac{x^3}{3}\right]_0^1 = \frac{1}{3}$$

微分方程:$$\frac{dy}{dx} = ky$$

\subsection{极限和级数}

极限:$$\lim_{x \to 0} \frac{\sin x}{x} = 1$$

无穷级数:$$\sum_{n=1}^{\infty} \frac{1}{n^2} = \frac{\pi^2}{6}$$

泰勒展开:$$e^x = \sum_{n=0}^{\infty} \frac{x^n}{n!}$$

\subsection{矩阵和行列式}

矩阵乘法:
\[
\begin{bmatrix}
a & b \\
c & d
\end{bmatrix}
\begin{bmatrix}
x \\
y
\end{bmatrix}
=
\begin{bmatrix}
ax + by \\
cx + dy
\end{bmatrix}
\]

行列式:
\[
\det
\begin{vmatrix}
a & b \\
c & d
\end{vmatrix}
= ad - bc
\]

\section{算法示例}

\begin{algorithm}
\caption{快速排序算法}
\begin{algorithmic}
\Require 数组 $A$,起始索引 $p$,结束索引 $r$
\Ensure 排序后的数组
\If{$p < r$}
    \State $q \gets \text{PARTITION}(A, p, r)$
    \State $\text{QUICKSORT}(A, p, q-1)$
    \State $\text{QUICKSORT}(A, q+1, r)$
\EndIf
\State \Return $A$
\end{algorithmic}
\end{algorithm}

\section{物理公式}

薛定谔方程:
\[
i\hbar\frac{\partial}{\partial t}\Psi(\mathbf{r},t) = \hat{H}\Psi(\mathbf{r},t)
\]

麦克斯韦方程组:
\begin{align*}
\nabla \cdot \mathbf{E} &= \frac{\rho}{\varepsilon_0} \\
\nabla \cdot \mathbf{B} &= 0 \\
\nabla \times \mathbf{E} &= -\frac{\partial \mathbf{B}}{\partial t} \\
\nabla \times \mathbf{B} &= \mu_0\mathbf{J} + \mu_0\varepsilon_0\frac{\partial \mathbf{E}}{\partial t}
\end{align*}

\section{概率统计}

正态分布概率密度函数:
\[
f(x) = \frac{1}{\sigma\sqrt{2\pi}} e^{-\frac{1}{2}\left(\frac{x-\mu}{\sigma}\right)^2}
\]

期望值:$E[X] = \int_{-\infty}^{\infty} x f(x) dx$

方差:$\text{Var}(X) = E[(X - \mu)^2]$

\section{结论}

本文档展示了 LaTeX 在数学公式渲染方面的强大能力。无论是简单的代数表达式还是复杂的微积分公式,LaTeX 都能提供清晰美观的排版效果。

\end{document}
